\documentclass[11pt]{article}
\usepackage[margin=0.74in]{geometry}
\geometry{letterpaper}
\usepackage{graphicx}
\usepackage{amssymb}
\usepackage{dejavu}
\usepackage{listings}
\usepackage{fancyhdr}
\pagestyle{fancy}

\chead{\scshape\bfseries Minesweeper Initial Design}
\rhead{Tristan Seifert, Blake Powell\\ CSCE 121-502}

\lstset{
	numbers=left,
	tabsize=4,
	numberfirstline=true,
	basicstyle=\footnotesize\ttfamily,
	breaklines=true
}

\begin{document}
\section{Overall Design}
The program will be divided into four parts:

\begin{itemize}
	\item \textbf{Main Window}: This is an FLTK window that contains all the widgets for the game, and handles all user input/game status.
	\item \textbf{Game Board}: A custom FLTK widget that draws a grid of cells, and contains the data for the bombs, flags, etc.
	\item \textbf{Persistence}: Stores the high scores in a file in the program directory, and loads them when the game is started again.
	\item \textbf{New Game Dialog}: Allows the user to select which difficulty level they want to play the game at, or use a custom grid size and mine count.
\end{itemize}

\section{Distribute Mines}
This algorithm will place $n$ mines randomly throughout the grid.

\begin{lstlisting}[frame=single,language=C++]
void generateMines(int n) {
	if(n > (this->gridW * this->gridH)) {
		throw runtime_error("cannot fill more mines than there are cells");
	}

	int minesPlaced = 0;

	while(minesPlaced < n) {
		// generate random X/Y coordinate
		int x = arc4random_uniform(this->gridW);
		int y = arc4random_uniform(this->gridH);

		// is there already a mine here?
		if(this->storage[x][y].isMine != true) {
			this->storage[x][y].isMine = true;
			minesPlaced++;
		}
	}
}
\end{lstlisting}

\section{Recursive Uncovering}
When a cell is uncovered with no surrounding mines, all cells adjacent to it (in an 3x3 square, with the clicked cell in the middle) are also uncovered, and so forth, until either all cells are uncovered, or a cell with an adjacent bomb is uncovered.


\begin{lstlisting}[frame=single,language=C++]
void uncoverCell(int x, int y, bool isRecursive = false) {
	Cell &t = this->storage[x][y];

	// check if this tile is flagged
	if(t.flagged == true) {
		return;
	}

	// check if this tile has been uncovered before
	if(this->uncoveredCells.contains(Point(x, y))) {
		return;
	}

	this->uncoveredCells.push_back(Point(x, y));

	// set uncovered flag
	t.uncovered = true;

	// is this a mine?
	if(t.isMine == true && !isRecursive) {
		// lmao game over
		this->_parent->gameOver();
		this->revealMines = true;

		t.bgRed = true;
	} else {
		// calculate how many mines there are around this point
		t.surroundingMines = this->minesAroundCell(x, y);

		// if clue value is zero, uncover all adjacent tiles
		if(t.surroundingMines == 0) {
			if((x - 1) >= 0 && (y - 1) >= 0) {
				this->uncoverCell((x - 1), (y - 1), true);
			}
			// top middle
			if(/*(x - 0) >= 0 &&*/ (y - 1) >= 0) {
				this->uncoverCell((x - 0), (y - 1), true);
			}
			// top right
			if((x + 1) < this->gridW && (y - 1) >= 0) {
				this->uncoverCell((x + 1), (y - 1), true);
			}

			// middle left
			if((x - 1) >= 0/* && (y - 1) >= 0*/) {
				this->uncoverCell((x - 1), (y - 0), true);
			}
			// middle right
			if((x + 1) < this->gridW/* && (y - 1) >= 0*/) {
				this->uncoverCell((x + 1), (y - 0), true);
			}

			// bottom left
			if((x - 1) >= 0 && (y + 1) < this->gridH) {
				this->uncoverCell((x - 1), (y + 1), true);
			}
			// bottom middle
			if(/*(x - 0) >= 0 &&*/ (y + 1) < this->gridH) {
				this->uncoverCell((x - 0), (y + 1), true);
			}
			// bottom right
			if((x + 1) < this->gridW && (y + 1) < this->gridH) {
				this->uncoverCell((x + 1), (y + 1), true);
			}
		}
	}
}
\end{lstlisting}


\section{Find Remaining Mines}
Determine the number of mines in the current grid. This is done by counting up all mines, then subtracting the number of flags from that count.

\begin{lstlisting}[frame=single,language=C++]
int getMinesRemaining() {
	int remaining = 0;

	// calculate the number of mines
	for(int i = 0; i < this->gridW; i++) {
		for(int j = 0; j < this->gridH; j++) {
			TileType t = this->storage[i][j];

			if(t.isMine == true) {
				remaining++;
			}
		}
	}

	// subtract the number of uncovered/flagged mines
	for(int i = 0; i < this->gridW; i++) {
		for(int j = 0; j < this->gridH; j++) {
			TileType t = this->storage[i][j];

			if(t.flagged == true || (t.isMine == true && t.uncovered == true)) {
				remaining--;
			}
		}
	}

	return remaining;
}
\end{lstlisting}

\section{Calculate Surrounding Mines}
Given a cell's $(x, y)$ coordinate, determine how many mines are surrounding it. (This is used to calculate the 'hint' value that's shown on the cells when they're uncovered.)

\begin{lstlisting}[frame=single,language=C++]
int minesAroundCell(int x, int y) {
	int mines = 0;
	if((x - 1) >= 0 && (y - 1) >= 0) {
		mines += (this->storage[(x - 1)][(y - 1)].isMine == true) ? 1 : 0;
	} if((y - 1) >= 0) {
		mines += (this->storage[(x - 0)][(y - 1)].isMine == true) ? 1 : 0;
	} if((x + 1) < this->gridW && (y - 1) >= 0) {
		mines += (this->storage[(x + 1)][(y - 1)].isMine == true) ? 1 : 0;
	} if((x - 1) >= 0) {
		mines += (this->storage[(x - 1)][(y - 0)].isMine == true) ? 1 : 0;
	} if((x + 1) < this->gridW) {
		mines += (this->storage[(x + 1)][(y - 0)].isMine == true) ? 1 : 0;
	} if((x - 1) >= 0 && (y + 1) < this->gridH) {
		mines += (this->storage[(x - 1)][(y + 1)].isMine == true) ? 1 : 0;
	} if((y + 1) < this->gridH) {
		mines += (this->storage[(x - 0)][(y + 1)].isMine == true) ? 1 : 0;
	} if((x + 1) < this->gridW && (y + 1) < this->gridH) {
		mines += (this->storage[(x + 1)][(y + 1)].isMine == true) ? 1 : 0;
	}
	return mines;
}
\end{lstlisting}

\section{Interface}
The same graphics as in the Windows XP desktop version of Minesweeper are used. See http://courses.cse.tamu.edu/jmichael/sp16/121/project/features.php (under the Board Visualization section,) or the second design document for images.

\end{document}  